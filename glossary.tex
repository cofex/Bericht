%gloassary and abbrevationfile

\newglossaryentry{foss}{
	name=floss,
	description={
	Free and open Software ist bezeichnet Software die gemeinhin als \enquote{Freie Software} bekannt ist. Dem Anwender werden die freiheit gegeben die Quelltexte der Anwendung zu studieren, die Anwendung zu modifizieren, mit anderen zu teilen und zu Nutzen wie er es will.}
	}
\newglossaryentry{geabi}{
	name=EaBI,
	description={Die \gls{eabi} ist eine spezielle \gls{abi} für embedded devices. Sie stellt unteranderem sicher, wenn ich z.B. eine library von einem Dritten nutzen muss, die mit dem proprietären Keil Compiler gebaut wurde, kann ich die in Binärerform in meine mit \gls{GCC} gebaute Firmware integrieren.(Nur technisch und nicht juristisch betrachtet!)}
}
\newglossaryentry{embdev}{
	name=embdedded Device,
	description={	eingebetteres gerät, häufig wird einfach das englische \enquote{embedded devices} benutzt.}
}
\newglossaryentry{binutils}{
	name=Binutils,
	description={Toolset Grundlegende Gnutools bereitstellt. wie linker, assembler}
}
\newglossaryentry{Copyleft}{
	name=Copyleft,
	description={ Copyleft ist ein Kunstwort und das gegenteil von restriktivrn Copxrights zu zeigen
	}
}
\newglossaryentry{baremetal}{
	name=baremetal Systeme,
	description={Im Vergleich zu heute gängigen PCs haben eingebettete Systeme oftmals kein Betriebssysten sondern die Application läuft direkt auf der Hardware. Das ist dann ein sogenannten baremetal System.}
	}

\newacronym{eabi}{EABI}{embedded application binary interface}
\newacronym{abi}{ABI}{application binary interface}
\newacronym{ocd}{OCD}{On Chip Debugger}
\newacronym{gcc}{gcc}{Gnu C Compiler}
\newacronym{mcu}{MCU}{Microcontroller Unit}

