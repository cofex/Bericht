\section{Lizenzen}

\enquote{Softwarelizenzen erklären, wie derjenige, der die Rechte an der Software hält (für gewöhnlich der Autor), sie verwendet sehen will und welche Freiheiten oder Einschränkungen sie besitzt. Ohne eine Lizenz könnten viele Verwendungsmöglichkeiten der Software verboten sein.}\footnote{http://fsfe.org/projects/ftf/faq-what-is-licensing.de.html}

\subsection{Freie Lizenzen}

Freie Lizenzen haben die Besonderheit dem Benutzer die vier Freiheiten die Software zu benutzen, zu studieren, zu modifieren und weiter zuverbreiten zu gewähren. Dadurch gehen für den Nutzer viele Vorteile einher. Unter anderem vermindert er das Risiko, dass seine Investition (Einrichtung, Schulung der Mitarbeiter) nicht verpufft, wenn der Hersteller sein Produkt einstellt oder Pleite geht.

Einige Freie Software Projekte wurden daher aus der Insolvenzmasse pleitegegangener Firmen herausgelöst oder wie Blender freigekauft\footnote{http://www.blender.org/blenderorg/blender-foundation/history/}.
Die bekanntesten \Gls{foss} Lizenzen sind die der GPL und BSD Familie. Da gibt es für die verschiedenen Freiheitsgrade diverse Abwandlungen und vor allem bei von den BSD-Lizenzen viele Modifikationen, die auf die Bedürfnisse der Autoren angepasst sind.

\subsection{Copyleft und Copyright}

Die wesentlichen Unterschiede wurden ja im Vortag beim Projekttreffen besprochen und die Art der Lizenz im Vorfeld schon festgestellt und im Treffen bestätigt.
Ob die Liziensierung unter einer BSD mit den Bibliotheken von Keil möglich ist, müsste aber nochmal geprüft werden.

Wenn man sich im klaren ist was man selber mit dem Programm machen will und was andere mit Kompilat und Quelltexte machen dürfen kann man sich auf eine spezielle Lizenz festlegen.

\subsection{Lizenzen im Projekt}

Um einen Vorschlag für eine passende Lizenz zu machen wurden die Quelltexte des Projekts untersucht.
Dabei wurden im groben die folgenden vier Quellen ausfindig gemacht.
\begin{itemize}
	\item vom Keil Compiler
	\item von Stepover
	\item aus Auftragsarbeit
	\item von der FH erstellt
\end{itemize}


\noindent Die Herrkunft der Quelltexte bringt natürich enormes Gewicht in den Auswahlprozess der Lizenzfindung.
Zum einen können schon einzelne Codezeilen das ganze Werk infizieren, zum anderen geben Lizenzen auch gewissen Anforderungen an das Projekt mit. Die gplartigen Lizenzen verlanden unter anderem, dass das Werk zu jederzeit compilier und ausführbar ist. Man kann also keine essentiellen Bibliotheken proprietärer Herrkunft benutzen.


\noindent Einige hardwarenahe Bibliotheken stammen direkt auf dem Pool des Keil Compilers und unterliegen einer einem eingeschränkten Nutzungsrecht, der im Lizenzvertrag oder EULA der IDE nachlesbar ist.


\noindent Über die Quelltexte von Stepover kann nur Stepover verfügen und über die aus den Auftragsarbeiten je nach Vertragsgestaltung auch Stepover oder der Auftragsprogrammierer.


\noindent FH intern kann nach Zustimmung der einzelnen Entwickler die Lizenz recht ungezwungen festgelegt werden. Man sollte nur beachten, dass man nach deutschen Recht kein Urheberschaft abtreten kann, sondern nur die Nutzungsrechte. Da im allgemeinen mit den Studenten keine (Arbeits)Verträge geschlossen wurden, in denen geregelt ist wer und in welchem Umfang die Nutzungsrechte an den erstellen Quelltexte hat bedarf es der Zustimmung des Autors.
