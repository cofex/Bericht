Im Rahmen des embeddedOptiBiohash Projektes im 5. Semester sahen wir uns neben der Hauptaufgabe mit einer Reihe an weiteren Fragestellungen konfrontriert, die teils gegeben wurden aber überwiegend selbst gestellt waren.

\noindent So haben wir uns unter anderem mit folgenden Fragen beschäftig:
\begin{itemize}
\item Welche Lizenstechnischen Freiheiten haben wir? z.B. um Quelltext aus anderen (freien) Bibliotheken zu nutzen
\item Gibt es Dokumentation zum Quelltext, bzw Minimalbeispiele um sich reinzufinden und mit der Plattform vertraut zu machen.
\item Gibt es alternativen zur proprietären IDE, die in zur Verfügung gestellten Evaluierungsversion nur recht kleine Firmare flashen kann, die ohne technische Einschränkungen und  auf richtigen Betriebssystemen funktioniert?
\end{itemize}

\noindent Dazu sollen wir noch, von den Betreuern gestellt, einige Tests zu Alterungsprozessen mit einem gegeben Datensatz und dem Biohash machen.


\noindent Im nachhinein Implementieren wir dann - die eigentliche Aufgabe- ein Konzept für eine mögliche graphische Oberfläche, die den Biohash auf dem Gerät nutzbar macht?


\noindent Die Dokumentation unterteilt sich gemäß der einzelnen Arbeitspakete in autarke Abschnitte, die auch losgelöst voneinander stehen können.

\begin{itemize}
\item  FOSS-Lizensen
\item  Fosstoolchain
\item  Alterungstest
\item  Implementierung
%\item  Ausblick/Resume
\end{itemize}
