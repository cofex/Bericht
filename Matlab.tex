\section{Matlab-Vergleiche}

F\"ur die Vergleiche der einzelnen Datens\"atze war es notwendig den vorhandenen Quellcode zu modifizieren. Anfangs stand also prim\"ar das Verst\"andnis f\"ur den Hashalgorithmus als solchen sowie dessen Implementierung in Matlab im Vordergrund. Dies nahm einige Zeit in Anspruch, da die Umsetzung des Algorithmus zwar mathematisch kompakt aber dadurch schwer verst\"andlich, implementiert worden ist. Als der Quellcode ausreichend modifiziert war fiel uns nach der ersten Versuchsreihe auf, dass die Werte unrealistisch hoch bis hin zu unm\"oglich waren. Indiz daf\"ur war, dass die Kollisionsrate und die Reproduktionsrate jedesmal bei 0 lag. Dies h\"atte eine Verifizierung unm\"oglich gemacht. Die Ursache daf\"ur war letztendlich, dass f\"ur die Simulation ein Toleranzvektor mit dem Wert 0 verwendet wurde. Um also realistischere Werte zu erzielen setzten wir diesen auf 1 mit dem Ergebniss, dass sich die Aussagekraft der Werte deutlich besserte. Nach der ersten Versuchsreihe fiel uns auf, dass die Ergebnis unrealistisch bis hin zu unm\"oglich waren. Indiz daf\"ur war dass die CR und die RR jedesmal bei 0 lag. Was eine Verifizierung unm\"oglich gemacht h\"atte. Ebenfalls war der Treshold zu hoch. Uns fiel uns dann auf, dass die Ursache hierf\"ur der Toleranzvektor war. Dieser war in der bisherigen Umsetzung immer auf 0. F\"ur das weitere Testverfahren nutzen wir dann einen Toleranzvektor von 1. Anschlie{\ss}end hatten die Ergebnisse auch eine wesentlich bessere Aussagekraft. 
Die besten Ergebnisse lieferten die Daten welche zeitnah aufgenommen und auch verifiziert wurden. Die Durchschnittswerte betrugen dabei:
\newline\newline
\centerline{$EER:$ 0,027950  $Treshold:$ 13,0561  $RR:$ 0,064906}
\newline\newline\noindent
Diese sind deutlich geringer als der Vergleich der Samples welche einen zeitlichen Abstand von 1-2 Monaten haben. Hierbei liegen die Durchschnittswerte bei: 
\newline\newline
\centerline{$EER:$ 0,104641  $Treshold:$ 19,8423  $RR:$ 0,010566}
\newline\newline\noindent
\"Ahnliche Werte wurden auch beim Vergleich j\"ungerer Referenzdaten mit \"alteren Verifikationsdaten (Tabelle \ref{tab:reverse}), \"alterer Referenzdaten mit neueren Referenzdaten (Tabelle \ref{tab:referenz}) und  \"alterer Verifikationsdaten mit neueren Verifikationsdaten (Tabelle \ref{tab:verifikation}). Zwischen der Aufnahme dereinzelnen Datens\"atze lag jeweils ein Zeitraum von einem Monat. Basierend auf diesen Daten haben wir dann bei der Implementation auf dem Ger\"at einen maximalen Treshold von 20 gew\"ahlt.